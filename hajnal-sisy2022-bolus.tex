\documentclass[conference]{IEEEtran}
\IEEEoverridecommandlockouts
% The preceding line is only needed to identify funding in the first footnote. If that is unneeded, please comment it out.
\usepackage{cite}
\usepackage{amsmath,amssymb,amsfonts}
\usepackage{algorithmic}
\usepackage{graphicx}
\usepackage{textcomp}
\usepackage{xcolor}

\def\BibTeX{{\rm B\kern-.05em{\sc i\kern-.025em b}\kern-.08em
    T\kern-.1667em\lower.7ex\hbox{E}\kern-.125emX}}
\begin{document}

\title{Experimental Bolus Sensor for Dairy Cattle\\
\thanks{Identify applicable funding agency here. If none, delete this.}
}

\author{\IEEEauthorblockN{Éva Nagyné Hajnal}
\IEEEauthorblockA{\textit{Alba Regia Technical Faculty} \\
\textit{Óbuda University}\\
Székesfehérvár, Hungary \\
email address or ORCID}
\and
\IEEEauthorblockN{Gergely Vakulya}
\IEEEauthorblockA{\textit{Alba Regia Technical Faculty} \\
\textit{Óbuda University}\\
Székesfehérvár, Hungary \\
email address or ORCID}
\and
\IEEEauthorblockN{Péter Udvardy}
\IEEEauthorblockA{\textit{Alba Regia Technical Faculty} \\
\textit{Óbuda University}\\
Székesfehérvár, Hungary \\
email address or ORCID}
}

\maketitle

\begin{abstract}
Abstract
\end{abstract}

\begin{IEEEkeywords}
dairy cattle, bolus, accelerometer
\end{IEEEkeywords}

\section{Introduction}

A great challenge of today's humanity is to serve the intensively
increasing demand for food, in an environmentally friendly manner.
A promising way to address this problem is precision agriculture,
based on extensive monitoring. In our research we focus on dairy
cattle and rumen bolus sensors. This paper presents the preliminary
results of an experimental bolus, equipped with accelerometer,
shock sensor and temperature meter.


%\begin{table}[htbp]
%\caption{Table Type Styles}
%\begin{center}
%\begin{tabular}{|c|c|c|c|}
%\hline
%\textbf{Table}&\multicolumn{3}{|c|}{\textbf{Table Column Head}} \\
%\cline{2-4} 
%\textbf{Head} & \textbf{\textit{Table column subhead}}& \textbf{\textit{Subhead}}& \textbf{\textit{Subhead}} \\
%\hline
%copy& More table copy$^{\mathrm{a}}$& &  \\
%\hline
%\multicolumn{4}{l}{$^{\mathrm{a}}$Sample of a Table footnote.}
%\end{tabular}
%\label{tab1}
%\end{center}
%\end{table}

\include{measurement_environment}

\section{Measurement setup}

\subsection{Bolus sensor}

\subsection{Gateways}

\subsection{Data format}

\begin{figure}[htbp]
\centerline{\includegraphics{fig/fig1.png}}
\caption{Example of a figure caption.}
\label{fig}
\end{figure}

\section{Results}

\cite{nagl2003}

\section{Summary}

\section*{Acknowledgment}

The preferred spelling of the word ``acknowledgment'' in America is without 
an ``e'' after the ``g''. Avoid the stilted expression ``one of us (R. B. 
G.) thanks $\ldots$''. Instead, try ``R. B. G. thanks$\ldots$''. Put sponsor 
acknowledgments in the unnumbered footnote on the first page.

\bibliographystyle{IEEEtran}
\bibliography{IEEEabrv,references}

\end{document}
